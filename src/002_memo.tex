\subsection{TCP/IP}
Transmission Control Protocol / Internet Protocol.
複数のプロトコル群の総称.

\begin{table}[htbp]
    \begin{center}
        \caption{OSI参照モデルとの対応}
        \begin{tabular}{ccc}
            OSI参照モデル & TCP/IP階層モデル & Protocol \\ \hline
            \sandan{アプリケーション層}{プレゼンテーション層}{セッション層} & アプリケーション層 & HTTP, SMTP, POP3, FTP,...\\ \hline
            トランスポート層 & トランスポート層 & TCP, UDP \\ \hline
            ネットワーク層 & ネットワーク層 & IP, ARP, ICMP, OSPF,... \\ \hline
            \nidan{データリンク層}{物理層} & ネットワークインターフェース層 & Ethernet, PPP, ...\\ \hline
        \end{tabular}
    \end{center}
\end{table}

\subsection{IP}
Internet Protocol.インターネット層で動作するプロトコル.主な特徴.
\begin{itemize}
    \item コネクションレス型通信
    \begin{itemize}
        \item コンピュータ間でコネクションを確立せずデータ伝送をはじめる
        \item TCPを使えばコンピュータ間ではコネクション型通信になる
    \end{itemize}
    \item ベストエフォート型通信
    \begin{itemize}
        \item 通信について,最大の努力は尽くすが品質は保証しない
        \item TCPを使えばパケット損失がないように見せられる
    \end{itemize}
    \item 階層型アドレッシング
    \begin{itemize}
        \item IPアドレス:IPプロトコルにより割り当てる論理アドレス
        \item 所属グループ(ネットワーク部),ネットワーク内での識別番号(ホスト部)から構成される
    \end{itemize}
\end{itemize}

\subsection{ARP}
Address Resolution Protocol.
IPアドレスからEthernetのMACアドレスを取得するプロトコル.
(セットで使うのに自動的な関係づけがない)

\begin{itemize}
    \item ARPの仕組み
    \begin{itemize}
        \item リクエスト:MACアドレスを知りたいノードのIPアドレスをブロードキャスト
        \item リプライ:自信のIPと一致したノードはユニキャストで返す
    \end{itemize}
\end{itemize}

\subsection{GARP}
Gratuitous ARP.役割は以下の2つ.
\begin{enumerate}
    \item 自身のIPアドレスが重複していないか検出
    \item ネットワーク機器上のARPキャッシュを更新
\end{enumerate}

\subsection{ARP table}
ARPの情報はARPキャッシュで一定時間保存.保存場所がARP table.
ARPはIPと同じネットワーク層で動作.ARPパケット.Ethernetフレームにカプセル化される.

\subsection{ICMP}
Internet Control Message Protocol.IPプロトコルのエラー通知・制御メッセージを転送するプロトコル.
通信状況の確認用.ネットワーク層で動作(IP層よりは上位)
メッセージの種類
\begin{enumerate}
    \item Query:問い合わせ
    \begin{itemize}
        \item 利用例:ping, tracerouteなど
    \end{itemize}
    \item Error:エラー通知
\end{enumerate}

\subsection{ping}
ICMPを利用したネットワーク診断.デフォルトで4回コマンドを送信する.
応答一覧
\begin{enumerate}
    \item Reply from:成功
    \item Request time out:失敗
    \begin{itemize}
        \item ケーブル接続されてない
        \item IPアドレス設定間違い
        \item Firewall有効
    \end{itemize}
    \item Destination net unreachable
    \begin{itemize}
        \item そもそも相手のnetworkに到達できてない
        \item ルータのアクセスリストでICMPが制限されている?
    \end{itemize}
    \item Destination host unreachable
    \begin{itemize}
        \item 相手のnetworkに到達したものの該当機器(host)に到達できない
        \item ルータのルーティングテーブル上にホストまでの経路情報がない
    \end{itemize}
    \item TTL expired in transit
    \begin{itemize}
        \item 宛先に到達するまでに128台ものルータを経由して途中でデータが捨てられてしまった
        \item 設定ミスによるルーティンググループが原因
    \end{itemize}
\end{enumerate}

\subsection{TCP}
Transmission Control Protocol.
IPの上位プロトコル.
トランスポート層で動作.
IPとセッション層以上(HTTP,FTP,Telnetなど)との橋渡し役.
信頼性の高い通信用(UDPは高速性重視)

\begin{enumerate}
    \item ポート番号を使用
    \begin{itemize}
        \item 通信先アプリケーションを特定する番号
        \begin{table}
            \centering
            \caption{port番号のリスト}
            \label{tab:tcp_port}
            \begin{tabular}{ccc}
                タイプ & 範囲 & 説明 \\ \hline
                well-known & 0~1023 & IANAで管理.サーバーのアプリケーション用 \\
                登録済み & 1024~49151 & IANAで管理.独自作成アプリケーション用 \\
                ダイナミック & 49152~65535 & クライアント側アプリケーション用 \\
            \end{tabular}
        \end{table}
    \end{itemize}
    \item コネクション確立・維持・切断
    \begin{itemize}
        \item 3way handshake(通信開始時)
        \begin{enumerate}
            \item こちらからの接続要求
            \item 相手からの接続確認
            \item こちらから接続応答
        \end{enumerate}
    \end{itemize}
    \item 順序制御
    \item 再送制御
    \item ウィンドウ制御
    \item フロー制御
\end{enumerate}

