\subsection{ネットワークとは}
\begin{enumerate}
    \item 規模によるnetworkの分類
    \begin{itemize}
        \item \detail{LAN}{Local Area Network}
        \begin{itemize}
            \item 同じ建物内の機器間でデータをやりとり
            \item 家・企業など
        \end{itemize}
        \item \detail{WAN}{Wide Area Network}
        \begin{itemize}
            \item 地理的に離れた機器間でデータをやりとり
            \item 本社と支店,大学のキャンパスネットワークなど
        \end{itemize}
        \item Internet
        \begin{itemize}
            \item 全世界レベルで相互接続したネットワーク
        \end{itemize}
        \item イントラネット
        \begin{itemize}
            \item Internet標準技術で構築された企業内ネットワーク
            \item 特定の組織内メンバーのみが閲覧可/メールの送受信可などの閉じたネット
            \item プロトコルにTCP/IPを利用した電子メール, Webシステムなど
            \item 外部接続する際はFire Wallなどを設置
        \end{itemize}
    \end{itemize}
\end{enumerate}

\subsection{ネットワークストレージとは}
ストレージの接続形態はDAS,NAS,SANの3種
\begin{enumerate}
    \item \detail{DAS}{Direct Attached Storage}
    \begin{itemize}
        \item 1台のサーバなどに直接接続する形態,またストレージそのもの
        \item 接続プロトコルはATA, SATA, SCSI, SAS
        \item 接続に専門知識不要
        \item 複数サーバでストレージ共有不可
    \end{itemize}
    \item \detail{NAS}{Network Attached Storage}
    \begin{itemize}
        \item ネットワークを介して接続する形態,またそのストレージそのもの
        \item HDD, LAN-Interface, OSなどで構成されるファイルサーバ専用機
        \item LAN接続型HDDとも
        \item ファイルシステムを通じてストレージを利用
    \end{itemize}
    \item \detail{SAN}{Storage Area Network}
    \begin{itemize}
        \item ストレージとサーバ間をFiber Channelで接続したネットワーク
        \item サーバと周辺機器が接続され高速伝送が可能
        \item LANから独立したストレージ専用のネットワーク
        \item データベース系含む幅広いアプリケーションに対応
        \item 導入コスト大,高度な専門知識が必須
        \item ストレージ装置そのものを共有する
    \end{itemize}
    \remark{EBMのDSUはSANなのか?}{SANはファイルシステムによる仲介を挟まない.そのため高速に動作するはず.またストレージの容量をフル活用できるはず.以上よりEBMのストレージとしてはSANが妥当なはず.}
\end{enumerate}

\subsection{Network Topology}
\begin{enumerate}
    \item 物理Topology(物理的な接続構成)
    \item 論理Topology(データの流れ方を表す構成)
    \item バス型Topology(物理)
    \begin{itemize}
        \item 1本の同軸ケーブルに複数ノードを接続
        \item 両端に抵抗器(Teminator)を付けて信号の反射・乱れを防ぐ
        \item 同軸ケーブルに障害発生→全部落ちる
    \end{itemize}
    \item スター型Topology(物理)
    \begin{itemize}
        \item ハブ(集線装置)に複数ノードを集約
        \item 各ケーブルの障害の影響はそこで止まる
        \item ハブに障害発生→全部落ちる
        \item ハブ同士を接続→拡張スター型
    \end{itemize}
    \item リング型Topology(論理)
    \begin{itemize}
        \item 全ノードを環状に接続
        \item リング上をデータが高速巡回
        \item 1か所障害発生→全部落ちる
        \item リングを増やして耐障害性向上→デュアルリング型
    \end{itemize}
    \item フルメッシュ型Topology
    \begin{itemize}
        \item 全ノードを相互接続
        \item どこが落ちても影響広がらない→耐障害性大
        \item 物理Topologyで実現→高コスト
        \item 中心や複数のルータのみをメッシュ化→Partial Mesh型
    \end{itemize}
    \remark{EBMのTopologyはPartial Mesh型?}{Mesh化によるコスト<<パターンミス等によるコスト.信頼性・耐障害性が最優先.装置内のクラスター規模はそこまで大きくない.FullMeshの可能性も?}
\end{enumerate}

\subsection{Protocol}